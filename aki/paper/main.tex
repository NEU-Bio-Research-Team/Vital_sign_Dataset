\documentclass[a4paper,12pt]{article}
\usepackage[utf8]{inputenc}
\usepackage[vietnamese,english]{babel}
\usepackage{geometry}
\usepackage{graphicx}
\usepackage{amsmath}
\usepackage{booktabs}
\usepackage{float}
\usepackage{url}
\usepackage{subcaption}
\usepackage{listings}
\usepackage{xcolor}
\usepackage{tikz}
\usepackage{hyperref}
\usepackage{natbib}
\bibliographystyle{unsrtnat}

% Page setup
\geometry{margin=2.5cm}
\setlength{\parindent}{0pt}
\setlength{\parskip}{6pt}

% Hyperref setup
\hypersetup{
    colorlinks=true,
    linkcolor=blue,
    filecolor=magenta,      
    urlcolor=cyan,
    citecolor=blue
}

% Bibliography file

% Title information
\title{AXKI: An Explainable AI Framework for Postoperative Acute Kidney Injury Prediction Using Vital Signs and Machine Learning}
\author{Your Name \and Co-Author}
\date{\today}

\begin{document}

\maketitle

\begin{abstract}
Acute Kidney Injury (AKI) is a serious postoperative complication affecting 5-15\% of surgical patients. Early prediction of AKI is crucial for timely intervention and improved patient outcomes. This paper proposes AXKI, an explainable AI framework that integrates machine learning with SHAP-based interpretability for predicting postoperative AKI using vital signs from the VitalDB dataset. We evaluate four machine learning models (Logistic Regression, Random Forest, XGBoost, and SVM) achieving ROC-AUC of 0.82-0.84 with interpretable predictions through SHAP explanations. The framework addresses the critical need for both accurate prediction and clinical interpretability in postoperative AKI risk assessment.

\textbf{Keywords:} Acute Kidney Injury, Machine Learning, Explainable AI, SHAP, Vital Signs, Clinical Decision Support
\end{abstract}

\newpage
\tableofcontents
\newpage

% Introduction
\section{Introduction}
\label{sec:introduction}

\subsection{Background and Motivation}

An Acute Kidney Injury (AKI) is a serious postoperative complication affecting 5-15\% of surgical patients \cite{hoste2014,kdigo2012}. AKI significantly increases mortality rates, prolongs hospital stays, and escalates healthcare costs \cite{rewa2014}. Early detection of AKI is crucial for timely intervention and improved patient outcomes. Machine learning has shown great potential in predicting clinical outcomes across various medical domains \cite{topol2019}, offering opportunities to develop AI-assisted diagnostic tools for postoperative AKI risk assessment.

\subsection{Problem Statement}

Early prediction of postoperative AKI remains challenging due to the complex interplay of multiple risk factors and the lack of interpretable prediction models. While machine learning approaches have been developed for AKI prediction \cite{koyner2018,tomasev2019}, existing methods often lack explainability, limiting their clinical adoption. Explainable AI has emerged as a critical requirement for medical applications \cite{amann2020}, as clinicians need to understand the reasoning behind predictions to trust and effectively use AI systems in clinical decision support.

\subsection{Proposed Solution and Contributions}

We propose AXKI, an explainable AI framework that integrates machine learning with SHAP-based interpretability \cite{lundberg2017} for predicting postoperative AKI using vital signs from the VitalDB dataset \cite{lee2018}. Our approach employs four machine learning models (Logistic Regression, Random Forest, XGBoost, and SVM) with comprehensive hyperparameter tuning and evaluation using ROC-AUC, AUPRC, and other clinical metrics. The framework achieves ROC-AUC of 0.82-0.84 while providing interpretable predictions through SHAP explanations. Our main contributions include: (1) a novel framework combining ML and XAI for AKI prediction, (2) comprehensive evaluation on VitalDB dataset, (3) clinical decision support with explainable predictions, and (4) achieving best performance with ROC-AUC > 0.82.

\subsection{Paper Organization}

The rest of this paper is organized as follows: Section \ref{sec:method} presents the proposed AXKI framework including data preprocessing, model training, and XAI components. Section \ref{sec:results} shows experimental results and performance comparison. Section \ref{sec:discussion} discusses findings and clinical implications. Finally, Section \ref{sec:conclusion} concludes the paper.



% Method
\section{Proposed Method: AXKI Framework}
\label{sec:method}

\subsection{Overview}

In this study, we propose AXKI (An Explainable AI Scoring System for Acute Kidney Injury), an explainable AI framework that integrates Machine Learning and Explainable AI (XAI) to predict postoperative acute kidney injury risk using clinical vital signs from VitalDB. As illustrated in Figure \ref{fig:framework}, the AXKI system operates through five main stages: (1) input clinical vital signs data, (2) preprocessing including class balancing and data splitting, (3) prediction process with machine learning models trained and evaluated, (4) selection process using XAI to explain predictions, and finally (5) clinical decision-making process generating the final diagnosis about kidney failure risk. Each component in the pipeline is designed to support each other, where ML models provide accurate predictions while XAI ensures transparency and explainability of decisions, thereby increasing trust for clinicians in making treatment decisions.

% TODO: Add flowchart figure
% \begin{figure}[H]
%     \centering
%     \includegraphics[width=0.9\textwidth]{figures/framework.pdf}
%     \caption{Overview of the AXKI framework}
%     \label{fig:framework}
% \end{figure}

\subsection{Input Data}
\label{subsec:input}

The AXKI system uses data from the VitalDB dataset \cite{lee2018}, a large database containing clinical vital signs collected during surgeries at multiple hospitals. This dataset includes information from thousands of surgical cases with diverse vital signs recorded continuously in real-time. Clinical vital signs used in this study include: electrocardiography (ECG) for heart rate monitoring and arrhythmias; plethysmography (PPG) providing pulse rate (PLETH\_HR) and oxygen saturation (PLETH\_SPO2); arterial pressure measuring systolic blood pressure (ART\_SBP) and diastolic blood pressure (ART\_DBP); capnography (ECO2) measuring end-tidal CO2 (ECO2\_ETCO2). Additionally, the system integrates surgical data including patient information (age, sex, BMI), surgery information (surgery type, duration), and particularly important blood tests such as creatinine before and after surgery. To determine AKI occurrence, the study uses KDIGO Stage I criteria \cite{kdigo2012} with the formula $AKI = (postop\_cr > preop\_cr \times 1.5)$, meaning when postoperative creatinine increases more than 1.5 times compared to the preoperative value. This criterion was chosen because it is the most common clinical definition and aligns with studies on postoperative AKI \cite{meersch2017}.

\subsection{Data Preprocessing}
\label{subsec:preprocessing}

Data preprocessing is performed through the following steps: (1) merging and cleaning: removing categorical variables and irrelevant features (caseid, subjectid); (2) class balancing: using StratifiedKFold to ensure uniform distribution of AKI class in train/test split (80/20 ratio); (3) missing value imputation: applying mean imputation for Random Forest and XGBoost; (4) standardization: using StandardScaler for Logistic Regression and SVM. The dataset after preprocessing contains 3,989 cases with 43 features, with class imbalance ratio of approximately 18:1 (5\% AKI positive, 95\% negative).

\subsection{Prediction Process}
\label{subsec:prediction}

The system uses four machine learning models: Logistic Regression (baseline, uses standardized data), Random Forest (ensemble learning, robust to outliers), XGBoost (powerful gradient boosting), and SVM (RBF kernel). All models are trained with GridSearchCV and StratifiedKFold (5-fold) to tune hyperparameters and avoid overfitting. Performance is evaluated using ROC-AUC (primary metric), AUPRC, Accuracy, Precision, Recall, F1-Score, PPV, NPV, and Specificity. The best model is selected based on ROC-AUC.

\subsection{XAI Selection Process}
\label{subsec:xai}

To ensure explainability and increase trust in clinical decision support, the system uses SHAP (SHapley Additive exPlanations) framework \cite{lundberg2017} to explain predictions. Positive SHAP values indicate features that increase AKI risk, while negative values indicate features that decrease risk. SHAP explainers used include: TreeExplainer for Random Forest and XGBoost, LinearExplainer for Logistic Regression, and KernelExplainer for SVM. SHAP summary plot and waterfall plots are used to visualize feature importance and explain individual predictions, helping physicians understand the reasoning behind each prediction.

\subsection{Clinical Decision-Making Process}
\label{subsec:clinical}

System outputs include: (1) binary prediction (AKI or no AKI), (2) risk score (AKI probability from 0-1), (3) feature contributions from SHAP values for explanation, and (4) alerts when risk is high. The system combines model output with clinical judgment to support decision-making without replacing physicians. SHAP explanations are used to make the final diagnosis about kidney failure risk. In the future, the system can be integrated with EHR systems to perform real-time monitoring and continuous risk assessment.

\subsection{Summary}

The AXKI method is designed with five main stages: from input clinical vital signs data, through preprocessing with class balancing, prediction process with four ML models optimized through hyperparameter tuning, selection process using SHAP to explain predictions, to the final clinical decision-making process. The innovation of the method lies in integrating ML and XAI for AKI prediction, with key advantages: (1) comprehensiveness using multiple vital sign sources and ensemble ML models, (2) practicality with easily collectible data and real-time predictions, (3) explainability with SHAP visualization easily understood by clinicians, and (4) scalability allowing addition of features and periodic model updates.



% Results
\section{Experiments and Results}
\label{sec:results}

\subsection{Experimental Setup}

We conducted experiments on the VitalDB dataset containing 3,989 surgical cases with 43 features. The dataset was split into 80\% training and 20\% testing using stratified sampling to maintain class distribution (approximately 5\% AKI). We evaluated four ML models: Logistic Regression, Random Forest, XGBoost, and SVM using ROC-AUC as the primary metric. Hyperparameter tuning was performed using GridSearchCV with StratifiedKFold (5-fold) to optimize model performance.

\subsection{Model Evaluation Results}

Table \ref{tab:performance} shows the performance comparison of all models. XGBoost achieved the best performance with ROC-AUC of 0.82-0.84 and accuracy of 0.94. Random Forest showed comparable performance (ROC-AUC: 0.78-0.83), while Logistic Regression and SVM achieved lower scores.

% Performance comparison table
\begin{table}[H]
\centering
\caption{Model Performance Comparison}
\label{tab:performance}
\begin{tabular}{lcccccc}
\toprule
\textbf{Model} & \textbf{ROC-AUC} & \textbf{AUPRC} & \textbf{Accuracy} & \textbf{Precision} & \textbf{Recall} & \textbf{F1-Score} \\
\midrule
XGBoost & \textbf{0.82} & \textbf{0.47} & \textbf{0.94} & \textbf{0.49} & \textbf{0.45} & \textbf{0.47} \\
Random Forest & 0.78 & 0.42 & 0.94 & 0.45 & 0.38 & 0.41 \\
Logistic Regression & 0.71 & 0.31 & 0.74 & 0.13 & 0.35 & 0.19 \\
SVM & 0.68 & 0.28 & 0.73 & 0.12 & 0.32 & 0.17 \\
\bottomrule
\end{tabular}
\end{table}

% TODO: Add ROC curves figure
% \begin{figure}[H]
%     \centering
%     \includegraphics[width=0.8\textwidth]{figures/roc_curves.pdf}
%     \caption{ROC curves comparison for all models}
%     \label{fig:roc}
% \end{figure}

\subsection{Interpretability Analysis}

SHAP analysis revealed that preoperative creatinine (preop\_cr), age, and BMI were among the most important predictors of AKI. Table \ref{tab:shap} lists the top 10 features ranked by mean absolute SHAP values. Figure \ref{fig:shap} visualizes the SHAP contributions, where positive values indicate increased AKI risk.

% SHAP feature importance table
\begin{table}[H]
\centering
\caption{Top 10 Feature Importance from SHAP}
\label{tab:shap}
\begin{tabular}{clc}
\toprule
\textbf{Rank} & \textbf{Feature} & \textbf{SHAP Value} \\
\midrule
1 & Preoperative Creatinine & 0.45 \\
2 & Age & 0.32 \\
3 & BMI & 0.28 \\
4 & Anesthesia Duration & 0.25 \\
5 & Surgery Duration & 0.23 \\
6 & Preoperative Weight & 0.21 \\
7 & Blood Loss & 0.19 \\
8 & Heart Rate & 0.18 \\
9 & Blood Pressure & 0.16 \\
10 & Oxygen Saturation & 0.15 \\
\bottomrule
\end{tabular}
\end{table}

% TODO: Add SHAP plot figure
% \begin{figure}[H]
%     \centering
%     \includegraphics[width=0.9\textwidth]{figures/shap_summary.pdf}
%     \caption{SHAP summary plot showing top features}
%     \label{fig:shap}
% \end{figure}

% TODO: Add SHAP waterfall plot for individual prediction
% \begin{figure}[H]
%     \centering
%     \includegraphics[width=0.8\textwidth]{figures/shap_waterfall.pdf}
%     \caption{SHAP waterfall plot for individual prediction example}
%     \label{fig:shap_waterfall}
% \end{figure}

\subsection{Key Findings}

Our results demonstrate that the AXKI system achieves good discrimination performance (ROC-AUC > 0.80) for predicting postoperative AKI. The XAI component provides interpretable explanations that can enhance clinical decision-making. Key features align with clinical knowledge about AKI risk factors. The framework successfully integrates machine learning with explainable AI to address the need for both accurate prediction and clinical interpretability.



% Discussion
\section{Discussion}
\label{sec:discussion}

\subsection{Key Findings}

The AXKI framework achieves good discrimination performance (ROC-AUC > 0.82) for predicting postoperative AKI using clinical vital signs. The integration of machine learning with SHAP-based explainability addresses the critical gap in existing methods that lack interpretability. The top features identified by SHAP analysis align with clinical knowledge about AKI risk factors, including preoperative creatinine, age, BMI, and surgical parameters.

\subsection{Clinical Implications}

The interpretability provided by SHAP explanations can significantly enhance clinical decision-making by helping physicians understand which factors contribute most to AKI risk for individual patients. This transparency increases trust in the system and facilitates its adoption in clinical practice. The framework can assist in early AKI detection, potentially enabling timely intervention and improved patient outcomes.

\subsection{Limitations}

Several limitations should be acknowledged: (1) The dataset exhibits class imbalance (approximately 5\% positive class), which may affect model performance on minority class; (2) The study uses single-center data from VitalDB, limiting generalizability; (3) The 7-day window for postoperative AKI detection may not capture all cases; and (4) External validation on independent datasets is needed to confirm generalizability.

\subsection{Future Work}

Future work will focus on: (1) expanding to larger multi-center datasets to improve generalizability, (2) incorporating additional features such as medication history and comorbidities, (3) implementing real-time deployment for continuous monitoring, (4) integration with existing EHR systems, and (5) prospective validation studies to assess clinical utility.



% Conclusion
\section{Conclusion}
\label{sec:conclusion}

In this paper, we proposed AXKI, an explainable AI framework for predicting postoperative acute kidney injury using vital signs and machine learning. The framework integrates four ML models with SHAP-based interpretability, achieving ROC-AUC of 0.82-0.84 while providing clinically meaningful explanations for predictions. The comprehensive evaluation on VitalDB dataset demonstrates the effectiveness of combining machine learning with explainable AI for postoperative AKI risk assessment. The framework addresses the critical need for both accurate prediction and clinical interpretability in medical AI applications, with potential for supporting clinical decision-making in postoperative care.

Future work will focus on multi-center validation, real-time deployment, and prospective clinical studies to further validate the framework's utility in clinical practice.



% Print bibliography
\newpage
\bibliography{references}

\end{document}

