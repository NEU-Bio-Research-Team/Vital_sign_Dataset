\section{Experiments and Results}
\label{sec:results}

\subsection{Experimental Setup}

We conducted experiments on the VitalDB dataset containing 3,989 surgical cases with 43 features. The dataset was split into 80\% training and 20\% testing using stratified sampling to maintain class distribution (approximately 5\% AKI). We evaluated four ML models: Logistic Regression, Random Forest, XGBoost, and SVM using ROC-AUC as the primary metric. Hyperparameter tuning was performed using GridSearchCV with StratifiedKFold (5-fold) to optimize model performance.

\subsection{Model Evaluation Results}

Table \ref{tab:performance} shows the performance comparison of all models. XGBoost achieved the best performance with ROC-AUC of 0.82-0.84 and accuracy of 0.94. Random Forest showed comparable performance (ROC-AUC: 0.78-0.83), while Logistic Regression and SVM achieved lower scores.

% Performance comparison table
\begin{table}[H]
\centering
\caption{Model Performance Comparison}
\label{tab:performance}
\begin{tabular}{lcccccc}
\toprule
\textbf{Model} & \textbf{ROC-AUC} & \textbf{AUPRC} & \textbf{Accuracy} & \textbf{Precision} & \textbf{Recall} & \textbf{F1-Score} \\
\midrule
XGBoost & \textbf{0.82} & \textbf{0.47} & \textbf{0.94} & \textbf{0.49} & \textbf{0.45} & \textbf{0.47} \\
Random Forest & 0.78 & 0.42 & 0.94 & 0.45 & 0.38 & 0.41 \\
Logistic Regression & 0.71 & 0.31 & 0.74 & 0.13 & 0.35 & 0.19 \\
SVM & 0.68 & 0.28 & 0.73 & 0.12 & 0.32 & 0.17 \\
\bottomrule
\end{tabular}
\end{table}

% TODO: Add ROC curves figure
% \begin{figure}[H]
%     \centering
%     \includegraphics[width=0.8\textwidth]{figures/roc_curves.pdf}
%     \caption{ROC curves comparison for all models}
%     \label{fig:roc}
% \end{figure}

\subsection{Interpretability Analysis}

SHAP analysis revealed that preoperative creatinine (preop\_cr), age, and BMI were among the most important predictors of AKI. Table \ref{tab:shap} lists the top 10 features ranked by mean absolute SHAP values. Figure \ref{fig:shap} visualizes the SHAP contributions, where positive values indicate increased AKI risk.

% SHAP feature importance table
\begin{table}[H]
\centering
\caption{Top 10 Feature Importance from SHAP}
\label{tab:shap}
\begin{tabular}{clc}
\toprule
\textbf{Rank} & \textbf{Feature} & \textbf{SHAP Value} \\
\midrule
1 & Preoperative Creatinine & 0.45 \\
2 & Age & 0.32 \\
3 & BMI & 0.28 \\
4 & Anesthesia Duration & 0.25 \\
5 & Surgery Duration & 0.23 \\
6 & Preoperative Weight & 0.21 \\
7 & Blood Loss & 0.19 \\
8 & Heart Rate & 0.18 \\
9 & Blood Pressure & 0.16 \\
10 & Oxygen Saturation & 0.15 \\
\bottomrule
\end{tabular}
\end{table}

% TODO: Add SHAP plot figure
% \begin{figure}[H]
%     \centering
%     \includegraphics[width=0.9\textwidth]{figures/shap_summary.pdf}
%     \caption{SHAP summary plot showing top features}
%     \label{fig:shap}
% \end{figure}

% TODO: Add SHAP waterfall plot for individual prediction
% \begin{figure}[H]
%     \centering
%     \includegraphics[width=0.8\textwidth]{figures/shap_waterfall.pdf}
%     \caption{SHAP waterfall plot for individual prediction example}
%     \label{fig:shap_waterfall}
% \end{figure}

\subsection{Key Findings}

Our results demonstrate that the AXKI system achieves good discrimination performance (ROC-AUC > 0.80) for predicting postoperative AKI. The XAI component provides interpretable explanations that can enhance clinical decision-making. Key features align with clinical knowledge about AKI risk factors. The framework successfully integrates machine learning with explainable AI to address the need for both accurate prediction and clinical interpretability.

